\documentclass[a4paper,12pt]{paper} % добавить leqno в [] для нумерации слева

%%% Работа с русским языком
\usepackage{cmap}					% поиск в PDF
\usepackage[T2A]{fontenc}			% кодировка
\usepackage[utf8]{inputenc}			% кодировка исходного текста
\usepackage[english,russian]{babel}	% локализация и переносы
\usepackage{indentfirst}

%%% Дополнительная работа с математикой
\usepackage{amsmath,amsfonts,amssymb,amsthm,mathtools} % AMS
\usepackage{icomma} % "Умная" запятая: $0,2$ --- число, $0, 2$ --- перечисление

%% Номера формул
\mathtoolsset{showonlyrefs=true} % Показывать номера только у тех формул, на которые есть \eqref{} в тексте.

%% Шрифты
\usepackage{euscript}	 % Шрифт Евклид
\usepackage{mathrsfs} % Красивый матшрифт

%% Свои команды
\DeclareMathOperator{\sgn}{\mathop{sgn}}

%% Перенос знаков в формулах (по Львовскому)
\newcommand*{\hm}[1]{#1\nobreak\discretionary{}
	{\hbox{$\mathsurround=0pt #1$}}{}}

\makeatletter
\def\@seccntformat#1{%
  \expandafter\ifx\csname c@#1\endcsname\c@section\else
  \csname the#1\endcsname\quad
  \fi}
\makeatother

%%% Заголовок

%%% Начало документа
\begin{document}
\section{Куприянов Артем. \\ Tricky Mutex [tricky] - Бонусная} 
\textbf{Решение: } \\
Эта реализация мьютекса не гарантирует свободу от взаимной блокировки. Приведем контр-пример:

Имеем 2 потока - A, B.
( == -- показывает чему равно значение переменной после текущего шага) \\
0) thread-count == 0 \\ 
1) \textbf{Поток А в lock делает fetch-add(1) и поток А зашел в критическую секцию.  }, thread-count == 1 \\
2) \textbf{Поток B в lock делает fetch-add(1). }, thread-count == 2 \\
3) \textbf{Поток А вышел из критической секции, делает unlock и fetch-sub(1). } thread-count == 1 \\
4) \textbf{Поток А делает lock и fetch-add(1)} thread-count == 2 \\
5) \textbf{Поток B делает в цикле while fetch-sub(1)} thread-count == 1 \\
6) \textbf{Поток B делает в цикле while fetch-add(1)} thread-count == 2 \\
7) \textbf{Поток B делает в цикле while fetch-sub(1)} thread-count == 1 \\ \\
\textbf{ Далее просто повторяем пункты 4, 5, 6, 7 и }

\textbf{Получим взаимную блокировку!!!}


Контр-пример работает!
\end{document}